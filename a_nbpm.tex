\chapter{The Analytical N-BPM method}

\begin{chapterinfo}
One of the most important figures of merit of optics correction is the deviation of the real $\beta$~function
from the model values,
the $\beta$~\emph{beating}. A deviation from the ideal values has a negative effect on machine performance.
Too high $\beta$~beating can even put machine components in danger and a threshold for operation with
physics filling scheme have to be imposed for machine protection reasons.

Furthermore, large $\beta$~beating deteriorates other linear and especially non-linear optics measurements
and correction methods. Therefore a good control of the $\beta$~beating can be essential for higher-order
correction steps.

This chapter summarises classical $\beta$~beating methods from turn-by-turn data and describes in
detail the improvements made by the \emph{analytical N-BPM method}.
\end{chapterinfo}

\section{$\beta$~function measurement from turn-by-turn data}

The classical method to measure the $\beta$~function in hadron machines is the \emph{Three BPM Method}
where we can make use of the transfer matrix element \eqref{eq_transfermatrix_beta_alpha}. By equating
the model and real transfer matrix elements we can infer the $\beta$~function from measured phase
and model values:
\begin{align}
    \frac{1}{\beta_i} \left( \cot\phi_{ij} + \alpha_i\right)
    &= \frac{1}{\beta\m_i} \left(\cot\phi_{ij}\m + \alpha_i\m\right)
    \label{eq_transfm12}\\
    \frac{1}{\beta_i} \left( \cot\phi_{ik} + \alpha_i\right)
    &= \frac{1}{\beta\m_i} \left(\cot\phi_{ik}\m + \alpha_i\m\right)
    \label{eq_transfm13}
\end{align}
Subtracting \eqref{eq_transfm13} from \eqref{eq_transfm12} and solving for $\beta_i$ yields
\begin{equation}
    \beta_i = \frac{\cot\phi_{ij} - \cot{\phi_{ik}}}{\cot\phi_{ij}\m - \cot{\phi_{ik}\m}}\beta_i\m
    \label{eq_three_bpm}
    \fstop
\end{equation}
This equation has two shortcomings: it assumes that the actual $\beta$~beating is very low and it
diverges for phase advances close to $n\pi$, $n \in \mathbb{N}$. The latter makes measurements at
exact $n\pi$ phase advances impossible and strongly enhances phase measurement errors near $n\pi$.
Unfortunately especially in the IR phase advances are close to 0 and precise $\beta$~function measurements
are not feasible.


\section{Original N-BPM method}

To avoid cases with unsuitable phase advances and to improve statistics, BPMs can be skipped and more combinations
can be used and averaged over with appropriate weights.
The N-BPM method was developped \cite{AndyNBPM, AndyThesis} to implement this feature.
\figureref{fig_three_nbpm} illustrates the principle. The Three BPM method takes three adjacent BPMs
for the calculation of the $\beta$~function. In any case there is at least one very unsuitable phase
advance involved. This renders the method unusable for LHC interaction regions.

\section{Corrected $\beta$ from phase formula}

\section{Error propagation}

\section{Quadrupolar like errors}

%\addtolength{\wrapoverhang}{0.5\marginparsep}
\begin{wrapfigure}{O}[1.7cm]{8cm}
	\centering
	\begin{tikzpicture}[x=1.2cm]
	%		\draw (-4,0) -- (-1,0);
	%		\draw (1,0) -- (4,0);
	%		\draw[fill=black!20!white] (-.5,-.5) rectangle (1,.5);
	%		\filldraw[fill=white, draw=black, dashed] (1,-.5) rectangle (1.5,.5);
	%		\draw[pattern=north west lines, pattern color=black!20!white] (-1,-.5) rectangle (-.5,.5);
	%		\draw[black] (-1,-.5) rectangle (1,.5);
	
	\draw (-2,0) -- (-1,0);
	\draw (1,0) -- (2,0);
	
	\filldraw[fill=white, draw=black, dashed] (1,-.5) rectangle (1.5,.5);
	\fill[white!80!black] (-1,-.5) rectangle (1,.5);
	\node at (0.0,0) {$ K_1 $};
	\draw[pattern=north west lines, draw=none]  (-1,-.5) rectangle (-.5,.5);
	\draw[dashed] (-1,-.5) rectangle (-.5,.5);
	\draw[black,thick] (-1,-.5) rectangle (1,.5);
	
	\draw[->] (-.5, .8) -- (-1, .8);
	\draw[dotted] (-.5,.5) -- (-.5, .9);
	\draw[dotted] (-1,.5) -- (-1, .9);
	\node[above] at (-.75, .8) {$ \delta s $};
	\end{tikzpicture}
	
	\vspace{.5cm}
	
	\begin{tikzpicture}[x=1.2cm]
	\draw (-2, 0) -- (-.5,0);
	\draw (1.5,0) --(2,0);
	%\draw (-.65,0) --(-.5,0);
	%\draw (1.0, 0) -- (1.15, 0);
	
	\filldraw[draw=black, thick, fill=white!80!black] (-.5,-.5) rectangle (1.5,.5);
	\node at (0.5,0) {$ K_1 $};
	\fill[white] (-.55, -.4) rectangle (-.45,.4);		
	\draw[pattern=north west lines] (-.55, -.4) rectangle (-.45,.4);
	
	\fill[white] (1.45, -.4) rectangle (1.55,.4);
	\draw[pattern=north east lines] (1.45, -.4) rectangle (1.55,.4);
	
	\node[below] at (-.5,-.5) {$k_1\delta s $};
	\node[below] at (1.5,-.5) {$- k_1\delta s  $};
	\end{tikzpicture}
	\caption{The top sketch shows the displaced quadrupole (solid gray) relative to the original position (dashed). In the bottom sketch one can see the quadrupole at its original position with thin magnets on both ends.}
	\label{fig:quadmisal}
\end{wrapfigure} 


\section{Routine $\beta$ measurement in LHC}

\section{HL-LHC simulations}

