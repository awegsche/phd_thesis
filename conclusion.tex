\chapter{Conclusion and Outlook}

The development and enhancement of three distinct optics measurement methods have been presented in this work.
These developments improve the precision and accuracy of optics measurements and extend our understanding of the
imperfections in an accelerator.

The first method, called the \emph{Analytical N-BPM method} is a more precise and faster measurement of the $\beta$~function.
It is based on the method previously developed and used for LHC measurements during run 1, called the N-BPM method,
that combines measurements from a set of $N$ neighbouring BPMs to remove statistical uncertainties.
Systematic uncertainties were taken into account through computationally expensive Monte-Carlo simulations.
The N-BPM method required a careful preparation for each individual optics setting before the actual measurement.
The analytical N-BPM method performs error propagation purely from analytic calculations and can therefore be applied
during the measurement without the need of precomputing the covariance matrix.
It also avoids the complications from failing simulations, which can happen for extreme optics settings,
like low $\beta^*$~optics for example.
A filtering of BPM combinations with unsuitable phase advances was also introduced in this method, which makes it even
more independent from the exact optics.
The analytical N-BPM method is now routinely used in various optics measurement and machine development
studies of various of CERN's main accelerators, including the LHC.
It has also been used in collaborative
efforts in accelerators from several external institutes like SuperKEKB and PETRA III.

The second method is a new local observable for linear lattice imperfections.
Although such observables already exists for non-linear imperfections, in the linear case there was none.
This work presents a local observable for the linear imperfections for the first time. This locality holds up to first
order in the quadrupole error $\delta K_1$.
The phase advance beating, which depends on all errors in the lattice, can be rearranged in a way that eliminates
global contributions, yielding two kinds of local observables:
if the unperturbed phase advance is a multiple of $\pi$, the phase advance beating is directly a local observable,
in the general case a combination of four nearby phase measurements has to be used to construct a local term.
The existence of this observable has been shown in this work.
Strong error sources can be detected using the local observable, which can be used to guide local error corrections.
Such a usage is planned for the upcoming run 3 of the LHC.

The last method describes the impact of forced particle motion on the measurement of transverse coupling.
The AC-dipole, which is used to create the forced motion for measurements, creates a jump in the amplitude of the
coupling RDTs. This jump does not appear in any of the current methods to calculate forced coupling and correct for
the forced motion.
In this work, forced coupling RDTs are calculated using a new framework that has recently been developed and which
accounts for this jump. The agreement for small coupling is excellent.
A future usage to compensate for the driven motion using an iterative approach is conceivable.

The introduction of next generation light sources and the design and construction of new colliders and future projects
create the demand for more advanced or novel measurement and correction methods.
More challenging optics designs require more precise measurements and large lattices call for efficient algorithms and
according implementation.
This work presents developments in the domain of linear optics measurements for circular accelerators,
improving existing methods and introducing new ones in order to prepare for future operation of the LHC and other accelerators.