\chapter{Advanced transverse coupling measurement}

\begin{chapterinfo}
    This chapter introduces methods to compensate for driven beam motion in the calculation of
    coupling resonance driving terms. 
\end{chapterinfo}

\section{Driven coupled motion}

\subsection{AC-dipole as skew quadrupole}

To start the section about the beam motion that is driven \emph{and} coupled, we state a curious
analogy. Throughout this chapter we use the following convention:
%
\begin{align}
    f_+ &\equiv f_{0101} \text{ and}\notag \\
    f_- &\equiv f_{0110} 
    \fstop
\end{align}
%
The RDTs in \eqref{eq_f0101_f0110} can be reformulated:
%
\begin{equation}
    f_\pm = 
    \frac{
        \sum\limits_w J_{1,w} \sqrt{\beta_{x,w}\beta_{y,w}}
        \e{
            i\left[\varphi_{wj,x} \pm \varphi_{wj,y}\right]
            -i\pi\left[Q_x\pm Q_y\right]
        }
    }{
        8i\sin[\pi(Q_x\pm Q_y)]
    }
    \fstop
\end{equation}
%
The coupled motion for a single coupling source now reads
%
\begin{align}
    h_x(s_j,N) =& \zxp(s_j,N) \notag \\
    &+2i\frac{
         J_{1,w} \sqrt{\beta_{x,w}\beta_{y,w}}
    }{
        8i\sin[\pi(Q_x + Q_y)]
    }
    \zyp(s_j,N)
        \e{
            i\left[\Delta\phi_x^+ + 2\pi Q_x \Theta(s_j,s_w) + 2\pi Q_y \Theta(s_j,s_w)\right]
            -i\pi\left[Q_x + Q_y\right]
        }
        \notag \\
    &+2i\frac{
         J_{1,w} \sqrt{\beta_{x,w}\beta_{y,w}}
    }{
        8i\sin[\pi(Q_x - Q_y)]
    }
    \zym(s_j,N)
        \e{
            i\left[\Delta\phi_x^- + 2\pi Q_x \Theta(s_j,s_w) - 2\pi Q_y \Theta(s_j,s_w)\right]
            -i\pi\left[Q_x - Q_y\right]
        }
        \komma
\end{align}
%
with 
%
\begin{equation}
    \Delta\phi_x^\pm = \varphi_x(s_j) - \varphi_x(s_w) \pm \left(\varphi_y(s_j) - \varphi_y(s_w)\right)
    \fstop
\end{equation}
%
The $\Theta(s_j,s_w)$ terms come from the wrapping around of the phase advance. Noting that 
$\zyp(s_j,N) = \sqrt{2I_y}\e{2\pi i N Q_y + \varphi(s_j)}$
one can bring this in a form similar to \eqref{eq_forced_motion}
%
\begin{align}
    h_x(s_j,N) =& \zxp(s_j,N) \notag \\
    &+\frac{
         J_{1,w} \sqrt{\beta_{x,w}\beta_{y,w}}
    }{
        4\sin[\pi(Q_x + Q_y)]
    }
    \sqrt{2I_y}
    \e{
        2\pi i N Q_y + \varphi(s_j)
        +i\left[\Delta\phi_x^+ + 2\pi (Q_x+Q_y) \text{sgn}(s_w-s_j)\right]
    }
        \notag \\
    &+\frac{
        J_{1,w} \sqrt{\beta_{x,w}\beta_{y,w}}
    }{
        4\sin[\pi(Q_x - Q_y)]
    }
    \sqrt{2I_y}
    \e{
        -2\pi i N Q_y + \varphi(s_j)
        +i\left[\Delta\phi_x^- + 2\pi (Q_x-Q_y) \text{sgn}(s_w-s_j)\right]
    }
    \fstop
\end{align}
%
The following table summarises which quantities get replaced
\begin{center}
\begin{tabular}{ll}
    AC dipole & coupling \\
    \hline
    \hline
    $ A_\theta $        & $ J_{1,w} $ \\
    $ \varphi_x(s_d) $  & $ \varphi_y(s_j) - \varphi_y(s_w) \mp \varphi_x(s_w)$ \\
    $ \Qd{x} $          & $ Q_y $\\
    \hline
\end{tabular}
\end{center}
Thus, a single coupling source acts like an AC dipole with the tune of the other plane as driving frequency.
From another perspective, the AC dipole couples the beam's motion to its oscillation.

\subsection{Derivation of the coupled driven motion}



To derive coupled driven motion, driven normal form coordinates, Eqs~(\ref{eq_driven_simpl2}) and
(\ref{eq_driven_simpl2_minus}) are inserted into \eqref{eq_coupled_h_from_z}. The driven normal form 
coordinates can easily be retrieved from the driven C-S coordinates:
%
\begin{align}
    \zeta_z^\pm
    &= \liemap{-F} = h_z^\pm + [-F, h_z^\pm] + O(2nd) \notag
    &= h
\end{align}
%

