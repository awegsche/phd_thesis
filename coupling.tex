\chapter{Advanced transverse coupling measurement}
\label{sec_coupling}

\begin{chapterinfo}
    This chapter introduces methods to compensate for driven beam motion in the calculation of
    coupling resonance driving terms. 
\end{chapterinfo}

\section{Driven coupled motion}

\subsection{AC-dipole as skew quadrupole}

To start the section about the beam motion that is driven \emph{and} coupled, we state a curious
analogy between AC-dipole and skew quadrupoles.

Throughout this chapter we use the following convention:
%
\begin{align}
    f_+ &\equiv f_{0101} \text{ and}\notag \\
    f_- &\equiv f_{0110} 
    \fstop
\end{align}
%
The RDTs in \eqref{eq_f0101_f0110} can be reformulated:
%
\begin{equation}
    f_\pm = 
    \frac{
        \sum\limits_w J_{1,w} \sqrt{\beta_{x,w}\beta_{y,w}}
        \e{
            i\left[\varphi_{wj,x} \pm \varphi_{wj,y}\right]
            -i\pi\left[Q_x\pm Q_y\right]
        }
    }{
        8i\sin[\pi(Q_x\pm Q_y)]
    }
    \fstop
\end{equation}
%
The coupled motion for a single coupling source now reads
%
\begin{align}
    h_x(s_j,N) =& \zxp(s_j,N) \notag \\
    &+2i\frac{
         J_{1,w} \sqrt{\beta_{x,w}\beta_{y,w}}
    }{
        8i\sin[\pi(Q_x + Q_y)]
    }
    \zyp(s_j,N)
        \e{
            i\left[\Delta\phi_x^+ + 2\pi Q_x \Theta(s_j,s_w) + 2\pi Q_y \Theta(s_j,s_w)\right]
            -i\pi\left[Q_x + Q_y\right]
        }
        \notag \\
    &+2i\frac{
         J_{1,w} \sqrt{\beta_{x,w}\beta_{y,w}}
    }{
        8i\sin[\pi(Q_x - Q_y)]
    }
    \zym(s_j,N)
        \e{
            i\left[\Delta\phi_x^- + 2\pi Q_x \Theta(s_j,s_w) - 2\pi Q_y \Theta(s_j,s_w)\right]
            -i\pi\left[Q_x - Q_y\right]
        }
        \komma
\end{align}
%
with 
%
\begin{equation}
    \Delta\phi_x^\pm = \varphi_x(s_j) - \varphi_x(s_w) \pm \left(\varphi_y(s_j) - \varphi_y(s_w)\right)
    \fstop
\end{equation}
%
The $\Theta(s_j,s_w)$ terms come from the wrapping around of the phase advance. Noting that 
$\zyp(s_j,N) = \sqrt{2I_y}\e{2\pi i N Q_y + \varphi(s_j)}$
one can bring this in a form similar to \eqref{eq_forced_motion}
%
\begin{align}
    h_x(s_j,N) =& \zxp(s_j,N) \notag \\
    &+\frac{
         J_{1,w} \sqrt{\beta_{x,w}\beta_{y,w}}
    }{
        4\sin[\pi(Q_x + Q_y)]
    }
    \sqrt{2I_y}
    \e{
        2\pi i N Q_y + \varphi(s_j)
        +i\left[\Delta\phi_x^+ + 2\pi (Q_x+Q_y) \text{sgn}(s_w-s_j)\right]
    }
        \notag \\
    &+\frac{
        J_{1,w} \sqrt{\beta_{x,w}\beta_{y,w}}
    }{
        4\sin[\pi(Q_x - Q_y)]
    }
    \sqrt{2I_y}
    \e{
        -2\pi i N Q_y + \varphi(s_j)
        +i\left[\Delta\phi_x^- + 2\pi (Q_x-Q_y) \text{sgn}(s_w-s_j)\right]
    }
    \fstop
\end{align}
%
The following table summarises which quantities get replaced
\begin{center}
\begin{tabular}{ll}
    AC dipole & coupling \\
    \hline
    \hline
    $ A_\theta $        & $ J_{1,w} $ \\
    $ \varphi_x(s_d) $  & $ \varphi_y(s_j) - \varphi_y(s_w) \mp \varphi_x(s_w)$ \\
    $ \Qd{x} $          & $ Q_y $\\
    \hline
\end{tabular}
\end{center}
Thus, a single coupling source acts like an AC dipole with the tune of the other plane as driving frequency.
From another perspective, the AC dipole couples the beam's motion to its oscillation.

\subsection{Derivation of the coupled driven motion}


The derivation of the coupled driven motion follows a similar path to the derivation of the 
coupled free motion but in the regime of driven motion, so the normal form approach has to be used.
Figure~\ref{fig_sketch_drv_ac} illustrates the procedure. The coordinates are propagated in normal form
space and transformed to physical space at $s_d-\epsilon$, directly in front of the AC-dipole. Then
an AC-dipole kick $\Delta h_x(N)$ is performed and the physical coordinate is transformed back to
normal form space where it is rotated around the ring. This process is repeated for each turn.

\begin{figure}
  \centering
  tikzpicture
      \caption{Kick performed in CS-coordinates, transformed to NF-coordinates}
  \label{fig_sketch_drv_ac}
\end{figure}

In the first turn, before the beam experiences the AC-dipole kick, the coordinates are those from
\eqref{eq_coupled_h_from_z}:
%
\begin{align}
  h_x^+(s_d-\epsilon, 0) &=  \e{\liemap{F}} \zeta_x^+(s_d-\epsilon, 0)\notag \\
  &=  \zeta^+(s_d-\epsilon,0)
    + 2i\conj{f_{1001}}\zeta_y^+(s_d-\epsilon, 0)
    + 2i\conj{f_{1010}}\zeta_y^-(s_d-\epsilon,0)
\end{align}
%
Then the particle is kicked in $p_x$ direction:
%
\begin{equation}
  h_x^+(s_d+\epsilon, 0) =  h_x^+(s_d - \epsilon,0) + \Delta h_x(0)
\end{equation}
%
For the transformation back to normal form space the inverse of \eqref{eq_h_from_z} has to be applied:
%
\begin{equation}
  \zeta = \e{\liemap{-F}} h = h + [-F,h] + O(h^3)
  \label{eq_coupled_h_after_kick}
\end{equation}
%
with
%
\begin{equation}
  F = \sum f_{jklm}\left( h_x^+ \right)^j \left( h_x^- \right)^k \left( h_y^+ \right)^l \left( h_y^- \right)^m
\end{equation}
%
when applied to $h_z^\pm$.
The normal form of the kicked particle motion now reads
%
\begin{align}
    \zeta_x^+(s_d+\epsilon, 0)
        &= h_x^+(s_d+\epsilon)
            - \conj{f_{1001}} \hyp(s_d+\epsilon, 0)
            - \conj{f_{1010}} \hym(s_d+\epsilon, 0)
    \notag\\
        &= \tilde{h}_x^+(s_d+\epsilon,0) + \Delta h_x(0)
            \notag \\ &\quad - \conj{f_{1001}} \left(\tilde{h}_y^+(s_d+\epsilon, 0) + \Delta h_y(0) \right)
            \notag \\ &\quad - \conj{f_{1010}} \left(\tilde{h}_y^-(s_d+\epsilon, 0) + \Delta h_y(0) \right)
    \notag \\
        &= \tilde{\zeta}_x^+(s_d+\epsilon, 0) + \Delta h_x(0)
            - \conj{f_{1001}} \Delta h_y(0)
            - \conj{f_{1010}} \Delta h_y(0)
\end{align}
%
where the tilde denotes undriven coordinates. The next step is to propagate the motion to the next
turn:
%
\begin{align}
    \zxp(s_d-\epsilon, 1) &= R_x\zxp(s_d+\epsilon, 0) \notag \\
        &= \tilde{\zeta}_x^+(s_d-\epsilon, 1) + R_x\Delta h_x(0)
            - \conj{f_{1001}} R_x\Delta h_y(0)
            - \conj{f_{1010}} R_x\Delta h_y(0)
\end{align}
%
Again, transformed to CS coordinates, before the second AC-dipole kick
%
\begin{align}
    \hxp(s_d-\epsilon, 1) &=R_x\zxp(s_d+\epsilon, 0) \notag \\
        &=
        \tilde{\zeta}_x^+(s_d-\epsilon, 1) + R_x\Delta h_x(0)
            - \conj{f_{1001}} R_x\Delta h_y(0)
            - \conj{f_{1010}} R_x\Delta h_y(0)
        \notag \\ &\quad 
            + \conj{f_{1001}} \left(\tilde{\zeta}_y^+(s_d-\epsilon, 1) + R_y\Delta h_y(0) \right)
            + \conj{f_{1010}} \left(\tilde{\zeta}_y^-(s_d-\epsilon, 1) + R_y\Delta h_y(0) \right)
    \fstop
    \label{eq_cpl_drv_n1}
\end{align}
%
In order to simplify this, the uncoupled driven coordinate from chapter \ref{sec_driven_coords} will
be introduced here as $h_x^{d\pm}$.
Now \eqref{eq_cpl_drv_n1} reads
%
\begin{equation}
    \hxp(s_d-\epsilon, 1)
        = h_x^{d+} + \conj{f_{1001}} h_y^{d+}(s_d+\epsilon, N) + \conj{f_{1010}} h_y^{d-}(s_d+\epsilon, N) 
            - \conj{f_{1001}} R_x\Delta h_y(0)
            - \conj{f_{1010}} R_x\Delta h_y(0)
    \fstop
\end{equation}
%
From this point it is easy to continue to arbitrary turn $N$:
%
\begin{align}
    \hxp(s<s_d, N) &= h_x^{d+}(s,N)
        + \conj{f_{1001}} h_y^{d+} (s, N)
        + \conj{f_{1010}} h_y^{d-} (s, N) \notag \\
        & \quad - \conj{f_{1001}} \sum\limits_{T = 1}^N R_x^{T}\Delta h_y(N-T) \notag \\
        & \quad - \conj{f_{1010}} \sum\limits_{T = 1}^N R_x^{T}\Delta h_y(N-T)
        \notag \\
    \hxp(s>s_d, N) &= h_x^{d+}(s,N)
        + \conj{f_{1001}} h_y^{d+} (s, N)
        + \conj{f_{1010}} h_y^{d-} (s, N) \notag \\
        & \quad - \conj{f_{1001}} \sum\limits_{T = 0}^N R_x^{T}\Delta h_y(N-T) \notag \\
        & \quad - \conj{f_{1010}} \sum\limits_{T = 0}^N R_x^{T}\Delta h_y(N-T)
\end{align}
%