\usepackage[dvipsnames]{xcolor}
\usepackage{tikz}
\usetikzlibrary{calc}
\usepackage{siunitx}

%\usepackage{xparse}
\pgfdeclarelayer{bg}    
\pgfsetlayers{bg,main}  

\definecolor{bpmused}{RGB}{225,120,0}
\definecolor{bpmnot}{RGB}{180,180,180}
\definecolor{bar}{RGB}{50,50,50}
\definecolor{quadblue}{RGB}{50,150,255}
\definecolor{sextclr}{RGB}{250,50,255}
\definecolor{otherclr}{RGB}{10,205,100}
\newcommand{\bpmthickness}{0.15}
\newcommand{\sextheight}{0.0}
\newcommand{\otherelemfactor}{4}
\newcommand{\drawbpm}[3]{
  \coordinate #3 at ($#1 + (0.4, 0)$);
  \fill[#2]  
  ($#1 + (-\bpmthickness, -.5)$)
  rectangle
  ($#1 + ( \bpmthickness,  .5)$);
}
\newcommand{\usedbpm}[3]{
  \drawbpm{#1}{bpmused}{#3};
  \node at #1 {#2};
}
\newcommand{\notusedbpm}[2]{
  \drawbpm{#1}{bpmnot}{#2}
}
\newcommand{\quadrupole}[2]{
  \fill[quadblue]
  ($#1 - (\bpmthickness, 0)$) --
  ($#1 - (0, \bpmthickness + .1)$) --
  ($#1 + (\bpmthickness, 0)$) -- 
  ($#1 + (0, \bpmthickness + .1)$);
  \coordinate #2 at ($#1 + (0.4, 0)$);
}
\newcommand{\tquadrupole}[2]{
  \quadrupole{#1}{#2}
  \node at #1 { {\tiny T} };
}
\newcommand{\otherelem}[2]{
  %\coordinate (Pcenter) at ($#1 + {(\otherelemfactor - 1)*\bpmthickness }*(1,0)$);
  %\fill[otherclr]
  %($(Pcenter) - (\otherelemfactor*\bpmthickness, 0)$) --
  %($(Pcenter) - (\otherelemfactor*\bpmthickness*.75, \bpmthickness + \sextheight)$) --
  %($(Pcenter) - (-\otherelemfactor*\bpmthickness*.75, \bpmthickness + \sextheight)$) --
  %($(Pcenter) + (\otherelemfactor*\bpmthickness, 0)$) -- 
  %($(Pcenter) + (\otherelemfactor*\bpmthickness*.75, \bpmthickness + \sextheight)$) -- 
  %($(Pcenter) + (-\otherelemfactor*\bpmthickness*.75, \bpmthickness + \sextheight)$);
  %\coordinate #2 at ($#1 + (0.4,0) + {(\otherelemfactor-1)*2*\bpmthickness}*(1,0) $);
  % do nothing, but create the next coordinate:
  \coordinate #2 at ($#1 + (0.4,0) + {(\otherelemfactor-1)*1*\bpmthickness}*(1,0) $);
}
\newcommand{\sextupole}[2]{
  \fill[sextclr]
  ($#1 - (\bpmthickness, 0)$) --
  ($#1 - (\bpmthickness*.5, \bpmthickness + \sextheight)$) --
  ($#1 - (-\bpmthickness*.5, \bpmthickness + \sextheight)$) --
  ($#1 + (\bpmthickness, 0)$) -- 
  ($#1 + (\bpmthickness*.5, \bpmthickness + \sextheight)$) -- 
  ($#1 + (-\bpmthickness*.5, \bpmthickness + \sextheight)$);
\coordinate #2 at ($#1 + (0.4, 0)$);
}

\newcommand{\printcomb}[1]
{
  \begin{tikzpicture}[y=.4cm,x=.8cm, baseline=-\the\dimexpr\fontdimen22\textfont2\relax]
    \foreach \b [count=\bi] in {#1}
    {
      \if\b -
        \notusedbpm{(\bi*.4,0)}{(a)};
      \else
        \usedbpm{(\bi*.4,0)}{$\b$}{(a)};
      \fi
    }
  \end{tikzpicture}
}

% --------------------------------------------------------------------------------------------------
% --- sketches -------------------------------------------------------------------------------------
% --------------------------------------------------------------------------------------------------

\newcommand{\divh}{0.1}
\newcommand{\drawdiv}[1]{
  \draw[thick] (#1,-\divh) -- (#1,\divh);
}
\newcommand{\drawphi}[5]{
  \draw (#2,#1) -- (#3,#1)
    node[above, midway]  {$\varphi_{#4#5}$};
  \draw[thick] ($ (#2,#1) + (0,\divh) $) -- ($ (#2,#1) - (0,\divh) $);
  \draw[thick] ($ (#3,#1) + (0,\divh) $) -- ($ (#3,#1) - (0,\divh) $);
  \draw[dotted]  (#2, 0) -- (#2, #1);
  \draw[dotted]  (#3, 0) -- (#3, #1);
}
\newcommand{\drawphaseline}[4]{
  \draw (0,0) -- (#4,0);
  \drawdiv{#1}
  \drawdiv{#2}
  \drawdiv{#3}
  \drawdiv{#4}
  \node[above] at ($(#1,\divh)$) {$i$};
  \node[above] at ($(#2,\divh)$) {$j$};
  \node[above] at ($(#3,\divh)$) {$k$};
  \node[above] at ($(#4,\divh)$) {$l$};
}
\newcommand{\drawphicomb}[4]{
  \drawphaseline{#1}{#2}{#3}{#4}
  \drawphi{-0.5}{#1}{#2}{i}{j};
  \drawphi{-1}{#1}{#3}{i}{k};
  \drawphi{-1.5}{#1}{#4}{i}{l};
}
\newcommand{\drawthetacomb}[4] {
  \drawphaseline{#1}{#2}{#3}{#4}
  \drawphi{-0.5}{#2}{#4}{j}{l};
  \drawphi{-1.0}{#2}{#3}{j}{k};
  \drawphi{-1.5}{#1}{#3}{i}{k};
  \drawphi{-2.0}{#1}{#4}{i}{l};
}
\newcommand{\drawphase}[2] {
  \draw ($(#1+0.9, 0.1)$) arc (45:135:0.5657)
  node[above, midway] {$\SI{#2}{\degree}$};
}

\newcommand{\mpad}{0.2}
\newcommand{\quadlength}{0.4}
\newcommand{\bpmlength}{0.1}
\newcommand{\bendlength}{2.75*\quadlength}
\newcommand{\magpad}{0.1}
\newcommand{\bpmpad}{0.025}
\newcommand{\fodolength}{10}

\newcommand{\fodoquad}{
  \filldraw[quadblue!50!black, fill=quadblue, thick] ($(A) + (\magpad, -\mpad)$) rectangle ($ (A) + (\quadlength + \magpad, .5 + \mpad)$);
  \coordinate (temp) at (A);
  \node at ($(A) + (\magpad + 0.5*\quadlength, 0.25)$) {{\tiny Q}};
  \coordinate (A) at ($(temp) + (\quadlength + 2.0*\magpad, 0)$);
}

\newcommand{\fodobend}{
  \filldraw[otherclr!50!black, fill=otherclr, thick] ($(A) + (\magpad, -\mpad)$) rectangle ($ (A) + (\bendlength + \magpad, 0.5 + \mpad)$);
  \node at ($(A) + (\magpad + 0.5*\bendlength, 0.25)$) {{\tiny bend}};
  \coordinate (temp) at (A);
  \coordinate (A) at ($(temp) + (\bendlength + 2.0*\magpad, 0)$);
}

\newcommand{\fodobpm}{
  \filldraw[bpmused!50!black, fill=bpmused, thick] ($(A) + (\bpmpad, -0.5*\mpad)$) rectangle ($ (A) + (\bpmlength + \bpmpad, 0.5*\mpad)$);
  \filldraw[bpmused!50!black, fill=bpmused, thick] ($(A) + (\bpmpad, 0.5-0.5*\mpad)$) rectangle ($ (A) + (\bpmlength + \bpmpad, 0.5+0.5*\mpad)$);
  \node[below] at ($(A) + (\bpmpad + 0.5*\bpmlength, -0.25)$) {{\tiny BPM}};
  \coordinate (temp) at (A);
  \coordinate (A) at ($(temp) + (\bpmlength + 2.0*\bpmpad, 0)$);
}

% --------------------------------------------------------------------------------------------------
% fig_lattice
% --------------------------------------------------------------------------------------------------

\newcommand{\cube}[3]{
    \tdplotsetrotatedcoords{-#3}{0}{0}
    %\begin{scope} [canvas is xy plane at z=0, cm={cos(#3),-sin(#3),sin(#3),cos(#3),(0,0)}]
    \filldraw[fill=gray, tdplot_rotated_coords]
        ($ #1 + (-#2,#2,-#2) $) -- ($ #1 + (#2,#2, -#2) $)
        -- ($ #1 + (#2,#2, #2,) $) -- ($ #1 + (-#2,#2, #2) $)
        -- ($ #1 + (-#2, #2,-#2) $) ;
    \filldraw[fill=gray, tdplot_rotated_coords]
        ($ #1 + (-#2,-#2,-#2) $) -- ($ #1 + (#2, -#2, -#2) $)
        -- ($ #1 + (#2, #2, -#2) $) -- ($ #1 + (-#2, #2, -#2) $)
        -- ($ #1 + (-#2,-#2,-#2) $) ;
    \filldraw[fill=gray, tdplot_rotated_coords]
        ($ #1 + (-#2,-#2,-#2) $) -- ($ #1 + (-#2,#2,-#2) $)
        -- ($ #1 + (-#2,#2, #2,) $) -- ($ #1 + (-#2,-#2,#2) $)
        -- ($ #1 + (-#2,-#2, #2,2) $) ;
    \filldraw[fill=gray, tdplot_rotated_coords]
        ($ #1 + (#2,-#2,-#2) $) -- ($ #1 + (#2,#2,-#2) $)
        -- ($ #1 + (#2,#2, #2,) $) -- ($ #1 + (#2,-#2,#2) $)
        -- ($ #1 + (#2,-#2, #2,2) $) ;
    \filldraw[fill=gray, tdplot_rotated_coords]
        ($ #1 + (-#2,-#2,#2) $) -- ($ #1 + (#2, -#2, #2) $)
        -- ($ #1 + (#2, #2, #2) $) -- ($ #1 + (-#2, #2, #2) $)
        -- ($ #1 + (-#2,-#2,#2) $) ;
    \filldraw[fill=gray, tdplot_rotated_coords]
        ($ #1 + (-#2,-#2,-#2) $) -- ($ #1 + (#2,-#2, -#2) $)
        -- ($ #1 + (#2,-#2, #2,) $) -- ($ #1 + (-#2,-#2, #2) $)
        -- ($ #1 + (-#2, -#2,-#2) $) ;
    %\end{scope}
}

\newcommand{\cuben}[3]{
    \tdplotsetrotatedcoords{-#3}{0}{0}
    %\begin{scope} [canvas is xy plane at z=0, cm={cos(#3),-sin(#3),sin(#3),cos(#3),(0,0)}]
    \filldraw[fill=gray, tdplot_rotated_coords]
        ($ #1 + (-#2,#2,-#2) $) -- ($ #1 + (#2,#2, -#2) $)
        -- ($ #1 + (#2,#2, #2,) $) -- ($ #1 + (-#2,#2, #2) $)
        -- ($ #1 + (-#2, #2,-#2) $) ;
    \filldraw[fill=gray, tdplot_rotated_coords]
        ($ #1 + (-#2,-#2,-#2) $) -- ($ #1 + (#2, -#2, -#2) $)
        -- ($ #1 + (#2, #2, -#2) $) -- ($ #1 + (-#2, #2, -#2) $)
        -- ($ #1 + (-#2,-#2,-#2) $) ;
    \filldraw[fill=gray, tdplot_rotated_coords]
        ($ #1 + (#2,-#2,-#2) $) -- ($ #1 + (#2,#2,-#2) $)
        -- ($ #1 + (#2,#2, #2,) $) -- ($ #1 + (#2,-#2,#2) $)
        -- ($ #1 + (#2,-#2, #2,2) $) ;
    \filldraw[fill=gray, tdplot_rotated_coords]
        ($ #1 + (-#2,-#2,-#2) $) -- ($ #1 + (-#2,#2,-#2) $)
        -- ($ #1 + (-#2,#2, #2,) $) -- ($ #1 + (-#2,-#2,#2) $)
        -- ($ #1 + (-#2,-#2, #2,2) $) ;
    \filldraw[fill=gray, tdplot_rotated_coords]
        ($ #1 + (-#2,-#2,#2) $) -- ($ #1 + (#2, -#2, #2) $)
        -- ($ #1 + (#2, #2, #2) $) -- ($ #1 + (-#2, #2, #2) $)
        -- ($ #1 + (-#2,-#2,#2) $) ;
    \filldraw[fill=gray, tdplot_rotated_coords]
        ($ #1 + (-#2,-#2,-#2) $) -- ($ #1 + (#2,-#2, -#2) $)
        -- ($ #1 + (#2,-#2, #2,) $) -- ($ #1 + (-#2,-#2, #2) $)
        -- ($ #1 + (-#2, -#2,-#2) $) ;
    %\end{scope}
}

\newcommand{\segment}[3]{
    
    \cube{(0,2,0)}{0.1}{#1};
    %\draw[red] ($ ({2.0*sin(#1)}, {2.0*cos(#1)}, 0) $) circle (0.1);
    \draw[thick] ($ ({2.0*sin(#1+3)}, {2.0*cos(#1+3)}, 0) $) arc (90-#1-3: 90-#2 : 2.0);
    \node[above, tdplot_rotated_coords] at (0,2,0.15) {$#3$};
}

\newcommand{\segmentn}[3]{
    \draw[thick] ($ ({2.0*sin(#1+3)}, {2.0*cos(#1+3)}, 0) $) arc (90-#1-3: 90-#2+3 : 2.0);
    \cuben{(0,2,0)}{0.1}{#1};
    %\draw[red] ($ ({2.0*sin(#1)}, {2.0*cos(#1)}, 0) $) circle (0.1);
    \node[above, tdplot_rotated_coords] at (0,2,0.15) {$#3$};
}
