\pagenumbering{roman}
\thispagestyle{empty}
\vspace{5cm}
\parskip0.75em
\begin{center}
    \Huge\bfseries
    Development of Optics Measurement Methods for Circular Accelerators

    \vspace{3cm}
    \mdseries\Large
    Dissertation\\
    \vspace{1em}
    zur Erlangung des Doktorgrads

    an der Fakult\"at f\"ur Mathematik, Informatik und Naturwissenschaften
    
    Fachbereich Physik
    
    \vspace{1em}
    der Universit\"at Hamburg

    \vspace{2cm}
    \vfill
    vorgelegt von\\
    Andreas Wegscheider
    \vspace{2em}

    Hamburg\\
    2022
\end{center}
\cleardoublepage
\newgeometry{
    inner=\innermarg,
    outer=\outermarg,
    top=3cm,
    bottom=3.5cm
}

\thispagestyle{empty}
\renewcommand\baselinestretch{1.5}\selectfont
\begin{tabular}{ll}
   Gutachter der Dissertation:%
   & Prof. Dr Wolfgang Hillert\\
   & Dr Andrea Franchi \vspace{1em} \\ 
   Mitglieder der Prüfungskommission:%
   & Prof. Dr Wolfgang Hillert\\
   & Dr Andrea Franchi\\
   & Prof. Dr Daniela Pfannkuche \\
   & Dr Frank Zimmermann\\
   & Prof. Dr Giuliano Franchetti \vspace{1.0em}  \\
   Vorsitzender der Prüfungskommission: & Prof. Dr Daniela Pfannkuche \vspace{1.0em}  \\
   Datum der Disputation: & 01.07.2022 \vspace{1.0em} \\
   Vorsitzender Fach-Promotionsausschuss PHYSIK: & Prof. Dr Wolfgang J. Parak \vspace{0.5em}  \\
   Leiter des Fachbereichs PHYSIK: & Prof. Dr Günter H. W. Sigl \vspace{0.5em}  \\
   Dekan der Fakultät MIN: & Prof. Dr Heinrich Graener
\end{tabular}

\vfill
\renewcommand\baselinestretch{\textlinespread}\selectfont
\noindent Dieses Dokument wurde mit LuaLaTeX und dem KOMA-Script gesetzt.\\
Hauptschriftart: STIX Two

\chapter*{Abstract}

Linear optics corrections in circular particle accelerators have achieved remarkable performance in the last years %
pushing the precision and accuracy of the measurement and correction of machine parameters further.
But development of accelerator technology is not resting either and the introduction of
next generation light sources and the design and construction of new colliders and future projects
constantly demand more advanced and precise measurement methods.

This work presents the development and enhancement of three distinct optics measurement methods.
The first one is a more precise, more accurate and faster measurement method of the $\beta$~function,
an optics parameter that presents a direct observable for the focusing at any given point in the machine.
Constraints on the tolerances of focusing errors are given for machine performance and protection reasons.
This work builds on an improvement presented in a previous work and further increases precision, accuracy and speed.
The second method is a novel local observable for linear lattice imperfections, which can be used to detect
strong error sources in the machine, guiding dedicated corrections and being independent of the optics configuration.
The last method provides a new way to describe the impact of forced particle motion on the measurement of transverse coupling.

The developments in the domain of linear optics measurements presented in this thesis
already positively impact LHC operation and machine development and are part of the preparation for
future operation of the LHC and other accelerators.


\chapter*{Zusammenfassung}
%\begin{german}
\foreignlanguage{ngerman}{
Im Gebiet der Korrektur linearer Optik in Ringbeschleunigern wurde in den letzten Jahren beachtenswerter Fortschritt gemacht.
Die Genauigkeit und Richtigkeit der Messung und Korrektur von Maschinenparametern wurde immer weiter verbessert.
Die Entwicklung von Beschleunigertechnologie hält jedoch nicht still und die Einführung von Lichtquellen der nächsten Generation
und Design und Bau von neuartigen Beschleunigern und Zukunftsprojekten erfordern neue, fortschrittliche Messmethoden.
}

\foreignlanguage{ngerman}{
Diese Arbeit stellt die Weiterentwicklung und Verbesserung von drei unterschiedlichen Optikmessmethoden vor.
Die erste Methode ist eine genauere und schnellere Messung der $\beta$-Funktion, einem Optikparameter,
der eine direkte Messgröße für die Fukussiereigenschaften an einem beliebigen Punkt im Beschleuniger darstellt.
Aus Gründen des Schutzes und der Leistungsfähigkeit der Maschine sind gewisse Anforderungen an die Fokussierung gegeben.
Die hier vorgestellte Messmethode baut auf einer vorangehenden Verbesserung der klassischen Methode zur Messung der $\beta$-Funktion auf
und liefert eine bessere Genauigkeit und k\:urzere Berechnungszeiten.
Die zweite Methode ist eine neue lokale Observable für lineare Maschinenfehler, die dazu benutzt werden kann, starke Fehlerquellen zu erkennen,
wodurch eine dedizierte Korrektur gezielt durchgeführt werden kann. Außerdem ist sie unabhängig von der
genauen Maschinenkonfiguration, wodurch die Ntwendigkeit der Messung jeder einzelnen Zwischenkonfiguration entf\:allt.
Die letzte Methode bietet eine neue Beschreibung des Effekts der getriebenen Schwingung der Teilchen auf die Messung der linearen transversalen Kopplung.
}

\foreignlanguage{ngerman}{
Die Weiterentwicklungen im Bereich der linearen Strahloptikmessung, die in dieser Arbeit vorgestellt werden,
erleichtern bereits den Betrieb des LHC und Maschinenentwicklungsstudien und sind Teil der Vorbereitungen für
den zukünftigen Betrieb des LHCs und anderer Beschleuniger.
}
%\end{german}

\cleardoublepage
\thispagestyle{empty}
\section*{Eidesstattliche Versicherung / Declaration on oath}

\foreignlanguage{german}{
Hiermit versichere ich an Eides statt, die vorliegende Dissertationsschrift selbst verfasst und
keine anderen als die angegeben Hilfsmittel und Quellen benutzt zu haben.\\[1.5em]
Die eingereichte schriftliche Fassung entspricht der auf dem elektronischen Speichermedium.\\[1.5em]
Die Dissertation wurde in der vorgelegten oder einer ähnlichen Form nicht schon einmal in einem früheren Promotionsverfahren angenommen oder als ungenügend beurteilt.
\vspace{6em}
}

\noindent Genf, den 02.03.2022
\hfill
\begin{tikzpicture}
    \draw (0,0) -- (5,0);
\end{tikzpicture}

%\begin{flushright}
\hfill \raisebox{1.0em}{Unterschrift des Doktoranden}
%\end{flushright}

\cleardoublepage
\section*{Acknowledgments}

It is not possible to arrive at this stage -- finishing a PhD project -- without the help and influence
of a considerable amount of people and chances are high that I will have forgotten to mention some of them.
This section will present my gratitude towards all the countless people who have been important to me
on my way to where I am now in a perceived chronological order.

Firstly I have to thank my fellow undergraduate students, especially \emph{Peter Freiwang}, \emph{Daniel Reiser} and
\emph{Alexis Kassiteridis} for treading the path with me and for showing me what is important
-- albeit by giving a negative example for a certain fellow student. Another person who must be
mentioned is Prof. Dr \emph{Harald Lesch} who not only awoke my interest in physics
when I was young but also turned out to be an inspiring professor, not only of Astrophysics but
especially in Philosphy.

Special thanks for arousing my interest in accelerator physics go to Prof. \emph{Lenny Rivkin} and
\emph{Adrian Oeftiger}.

I want to thank my colleagues and friends here at CERN for the memorable time, both in the office and
in our free time: \emph{Marco D'Andrea}, \emph{Sondre Vik Furuseth},
\emph{Alexander Krainer}, \emph{Elena Fol} and \emph{Joschua Dilly}.
Some deserve a special mention:\\
%\begin{itemize}
%\item
\emph{Jaime Coello de Portugal} (and I skip the rest of his name) for an enthusiastic introduction
to programming, computer games and for pulling me into some fun hobby projects like soldering my own
keyboard.\\
%\item
\emph{Michael Hofer} for being at the same time an invaluably helpfull and incredibly annoying
office mate, for his view of the bigger picture, his bad jokes and the tea and cookie times.\\
And to \emph{Marian Lückhof} for the many many dicussions and for preferring early lunch.
%\end{itemize}

And last but not least I want to express my deep gratitude to my family: My parents, \emph{Martin}
and \emph{Sigrid} who raised me to become the person I am now, for believing in me and supporting me.
And of course to \emph{Lorène} for being in my life.


\tableofcontents
\cleardoublepage
%\thispagestyle{plain}
